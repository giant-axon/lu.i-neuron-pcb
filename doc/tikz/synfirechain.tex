\documentclass[crop,tikz]{standalone}

\usepackage{svg}

\usetikzlibrary{shapes.geometric}
\usetikzlibrary{arrows.meta}
\usetikzlibrary{positioning}
\usetikzlibrary{calc}

\usepackage[outline]{contour}

\usepackage{rotating}

\begin{document}


\definecolor{pcbcolour}{RGB}{90, 32, 90}

\newcommand{\pcbNeuron}[5]{
    % macro for sketch used in overviews
    % arg1: name of nodes
    % arg2: x
    % arg3: y
    % arg4: scale
    % arg5: rotation
    % %%%%%%%%%%%%% draw stuff
    \def\x{#2};
    \def\y{#3}

    \ifthenelse{ \equal{#4}{} }{
        \def\scale{1}
    }{
        \def\scale{#4}
    }
    \ifthenelse{ \equal{#5}{} }{
        \def\rot{0}
    }{
        \def\rot{#5}
    }

    \coordinate (#1) at (\x, \y);

    \node[anchor=center,
          xshift=0.25cm * \scale,yshift=0.1cm * \scale, 
          rotate around={\rot:(#1)},
          ] (whitehead) at (#1)
    {
        % \scaledWidth
        \scalebox{\scale}{
            \includesvg[width=2cm]{neuron_filled.svg}
        }
    };

    % %%%%%%%%%%%%%%%%%%% dendrites
    \pgfmathsetmacro{\lengthDaX}{-0.003cm * \scale}
    \pgfmathsetmacro{\lengthDaY}{0.022cm * \scale}
    \coordinate[rotate around={\rot:(#1)}] (#1-Da) at (\x + \lengthDaX, \y + \lengthDaY);

    \pgfmathsetmacro{\lengthDbX}{-0.023cm * \scale}
    \pgfmathsetmacro{\lengthDbY}{0.005cm * \scale}
    \coordinate[rotate around={\rot:(#1)}]  (#1-Db) at (\x + \lengthDbX, \y + \lengthDbY);

    \pgfmathsetmacro{\lengthDcX}{-0.016cm * \scale}
    \pgfmathsetmacro{\lengthDcY}{-0.0145cm * \scale}
    \coordinate[rotate around={\rot:(#1)}]  (#1-Dc) at (\x + \lengthDcX, \y + \lengthDcY);

    % %%%%%%%%%%%%%%%%%%% axons
    \pgfmathsetmacro{\lengthAxX}{0.0400cm * \scale}
    \pgfmathsetmacro{\lengthAxY}{-0.0130cm * \scale}
    \coordinate[rotate around={\rot:(#1)}]  (#1-Aa) at (\x + \lengthAxX, \y + \lengthAxY);
    \pgfmathsetmacro{\lengthAxX}{0.0395cm * \scale}
    \pgfmathsetmacro{\lengthAxY}{-0.0140cm * \scale}
    \coordinate[rotate around={\rot:(#1)}]  (#1-Ab) at (\x + \lengthAxX, \y + \lengthAxY);
    \pgfmathsetmacro{\lengthAxX}{0.0390cm * \scale}
    \pgfmathsetmacro{\lengthAxY}{-0.0150cm * \scale}
    \coordinate[rotate around={\rot:(#1)}]  (#1-Ac) at (\x + \lengthAxX, \y + \lengthAxY);
}


\begin{tikzpicture}[
    scale=1.,
    >=latex,
    transform shape,
]
    \useasboundingbox (-5, -4) rectangle (5, 4);

    \def\numNeurons{7}
    \def\axlengthA{4.2}
    \def\axlengthB{3.0}
    \def\radius{4}
    \def\scale{1.0}
    \def\halfAngleConnection{10}


    % % draw ellipse on which the neurons lie
    % \draw (0,0) ellipse (\axlengthA cm and \axlengthB cm );

    % neurons
    \foreach \i in {1,...,\numNeurons}
    {
        \pgfmathsetmacro\usedAngle{360 * \i / \numNeurons }

        \pcbNeuron{neuron-\i}{
            % [rotate around={\usedAngle:(0, 0)}] \radius, 0  % circle
            $(0,0)+(\usedAngle:\axlengthA cm and \axlengthB cm)$  % ellipse
        }{\scale}{\usedAngle + 290} ;
    }

    % connections
    \foreach \i in {1,...,\numNeurons}
    {
        \pgfmathtruncatemacro\nexts{ifthenelse(\i==\numNeurons, 1, round(\i+1))}
        \pgfmathsetmacro\angleOut{360/\numNeurons * (\i-1) + 180 - \halfAngleConnection}
        \pgfmathsetmacro\angleIn{360/\numNeurons * (\i) - 40 + \halfAngleConnection}
        \draw[connection] (neuron-\i-Db) to[out=\angleOut,in=\angleIn] (neuron-\nexts-Ab);
    }


\end{tikzpicture}
\end{document}
