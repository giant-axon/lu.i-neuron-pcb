\documentclass[crop,tikz]{standalone}

\usepackage{svg}

\usetikzlibrary{shapes.geometric}
\usetikzlibrary{arrows.meta}
\usetikzlibrary{positioning}
\usetikzlibrary{calc}

\usepackage[outline]{contour}

\usepackage{rotating}

\begin{document}


\definecolor{pcbcolour}{RGB}{90, 32, 90}

\newcommand{\pcbNeuron}[4]{
    % macro for sketch used in overviews
    % arg1: name of nodes
    % arg2: point of node 
    % arg3: argScale
    % arg4: rotation
    % %%%%%%%%%%%%% draw stuff

    \ifthenelse{ \equal{#3}{} }{
        \def\argScale{1}
    }{
        \def\argScale{#3}
    }
    \ifthenelse{ \equal{#4}{} }{
        \def\rot{0}
    }{
        \def\rot{#4}
    }

    \coordinate (#1) at (#2);

    \node[anchor=center,
          xshift=0.25cm * \argScale,yshift=0.1cm * \argScale, 
          rotate around={\rot:(#1)},
          ] (whitehead) at (#1)
    {
        % \scaledWidth
        \scalebox{\argScale}{
            \includesvg[width=2cm]{neuron_filled.svg}
        }
    };

    % %%%%%%%%%%%%%%%%%%% dendrites
    \pgfmathsetmacro{\lengthDaX}{-0.10cm * \argScale}
    \pgfmathsetmacro{\lengthDaY}{0.62cm * \argScale}
    \coordinate[rotate around={\rot:(#1)}] (#1-Da) at ([xshift=\lengthDaX, yshift=\lengthDaY] #1);

    \pgfmathsetmacro{\lengthDbX}{-0.68cm * \argScale}
    \pgfmathsetmacro{\lengthDbY}{0.13cm * \argScale}
    \coordinate[rotate around={\rot:(#1)}]  (#1-Db) at ([xshift=\lengthDbX, yshift=\lengthDbY] #1);

    \pgfmathsetmacro{\lengthDcX}{-0.47cm * \argScale}
    \pgfmathsetmacro{\lengthDcY}{-0.42cm * \argScale}
    \coordinate[rotate around={\rot:(#1)}]  (#1-Dc) at ([xshift=\lengthDcX, yshift=\lengthDcY] #1);

    % %%%%%%%%%%%%%%%%%%% axons
    \pgfmathsetmacro{\lengthAaX}{1.150cm * \argScale}
    \pgfmathsetmacro{\lengthAaY}{-0.360cm * \argScale}
    \coordinate[rotate around={\rot:(#1)}]  (#1-Aa) at ([xshift=\lengthAaX, yshift=\lengthAaY] #1);

    \pgfmathsetmacro{\lengthAbX}{1.135cm * \argScale}
    \pgfmathsetmacro{\lengthAbY}{-0.390cm * \argScale}
    \coordinate[rotate around={\rot:(#1)}]  (#1-Ab) at ([xshift=\lengthAbX, yshift=\lengthAbY] #1);

    \pgfmathsetmacro{\lengthAcX}{1.120cm * \argScale}
    \pgfmathsetmacro{\lengthAcY}{-0.420cm * \argScale}
    \coordinate[rotate around={\rot:(#1)}]  (#1-Ac) at ([xshift=\lengthAcX, yshift=\lengthAcY] #1);
}


\begin{tikzpicture}[
    scale=1.,
    >=latex,
    transform shape,
]
    \useasboundingbox[fill=white] (-7, -4) rectangle (7, 4);

    \def\numNeurons{7}
    \def\axlengthA{4.2}
    \def\axlengthB{3.0}
    \def\radius{4}
    \def\scale{1.0}
    \def\halfAngleConnection{10}


    % % draw ellipse on which the neurons lie
    % \draw (0,0) ellipse (\axlengthA cm and \axlengthB cm );

    % neurons
    \foreach \i in {1,...,\numNeurons}
    {
        \pgfmathsetmacro\usedAngle{360 * \i / \numNeurons }

        \pcbNeuron{neuron-\i}{
            % [rotate around={\usedAngle:(0, 0)}] \radius, 0  % circle
            $(0,0)+(\usedAngle:\axlengthA cm and \axlengthB cm)$  % ellipse
        }{\scale}{\usedAngle + 290} ;
    }

    % connections
    \foreach \i in {1,...,\numNeurons}
    {
        \pgfmathtruncatemacro\nexts{ifthenelse(\i==\numNeurons, 1, round(\i+1))}
        \pgfmathsetmacro\angleOut{360/\numNeurons * (\i-1) + 180 - \halfAngleConnection}
        \pgfmathsetmacro\angleIn{360/\numNeurons * (\i) - 40 + \halfAngleConnection}
        \draw[connection] (neuron-\i-Db) to[out=\angleOut,in=\angleIn] (neuron-\nexts-Ab);
    }


\end{tikzpicture}
\end{document}
