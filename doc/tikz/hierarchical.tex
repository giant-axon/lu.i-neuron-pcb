\documentclass[crop,tikz]{standalone}

\usepackage{svg}

\usetikzlibrary{shapes.geometric}
\usetikzlibrary{arrows.meta}
\usetikzlibrary{positioning}
\usetikzlibrary{calc}

\usepackage[outline]{contour}

\usepackage{rotating}

\begin{document}


\definecolor{pcbcolour}{RGB}{90, 32, 90}

\newcommand{\pcbNeuron}[4]{
    % macro for sketch used in overviews
    % arg1: name of nodes
    % arg2: point of node 
    % arg3: argScale
    % arg4: rotation
    % %%%%%%%%%%%%% draw stuff

    \ifthenelse{ \equal{#3}{} }{
        \def\argScale{1}
    }{
        \def\argScale{#3}
    }
    \ifthenelse{ \equal{#4}{} }{
        \def\rot{0}
    }{
        \def\rot{#4}
    }

    \coordinate (#1) at (#2);

    \node[anchor=center,
          xshift=0.25cm * \argScale,yshift=0.1cm * \argScale, 
          rotate around={\rot:(#1)},
          ] (whitehead) at (#1)
    {
        % \scaledWidth
        \scalebox{\argScale}{
            \includesvg[width=2cm]{neuron_filled.svg}
        }
    };

    % %%%%%%%%%%%%%%%%%%% dendrites
    \pgfmathsetmacro{\lengthDaX}{-0.10cm * \argScale}
    \pgfmathsetmacro{\lengthDaY}{0.62cm * \argScale}
    \coordinate[rotate around={\rot:(#1)}] (#1-Da) at ([xshift=\lengthDaX, yshift=\lengthDaY] #1);

    \pgfmathsetmacro{\lengthDbX}{-0.68cm * \argScale}
    \pgfmathsetmacro{\lengthDbY}{0.13cm * \argScale}
    \coordinate[rotate around={\rot:(#1)}]  (#1-Db) at ([xshift=\lengthDbX, yshift=\lengthDbY] #1);

    \pgfmathsetmacro{\lengthDcX}{-0.47cm * \argScale}
    \pgfmathsetmacro{\lengthDcY}{-0.42cm * \argScale}
    \coordinate[rotate around={\rot:(#1)}]  (#1-Dc) at ([xshift=\lengthDcX, yshift=\lengthDcY] #1);

    % %%%%%%%%%%%%%%%%%%% axons
    \pgfmathsetmacro{\lengthAaX}{1.150cm * \argScale}
    \pgfmathsetmacro{\lengthAaY}{-0.360cm * \argScale}
    \coordinate[rotate around={\rot:(#1)}]  (#1-Aa) at ([xshift=\lengthAaX, yshift=\lengthAaY] #1);

    \pgfmathsetmacro{\lengthAbX}{1.135cm * \argScale}
    \pgfmathsetmacro{\lengthAbY}{-0.390cm * \argScale}
    \coordinate[rotate around={\rot:(#1)}]  (#1-Ab) at ([xshift=\lengthAbX, yshift=\lengthAbY] #1);

    \pgfmathsetmacro{\lengthAcX}{1.120cm * \argScale}
    \pgfmathsetmacro{\lengthAcY}{-0.420cm * \argScale}
    \coordinate[rotate around={\rot:(#1)}]  (#1-Ac) at ([xshift=\lengthAcX, yshift=\lengthAcY] #1);
}


\begin{tikzpicture}[
    scale=1.,
    >=latex,
    transform shape,
]
    \useasboundingbox (-5, -4) rectangle (5, 4);

    \def\numNeuronsI{2}
    \def\numNeuronsH{3}
    \def\numNeuronsO{2}
    \def\scale{1.0}

    \def\posXI{-4}
    \def\posXH{-0.25}
    \def\posXO{3.5}

    \def\lengthSeparation{2.8}

    \def\angleNeuron{10}
    \def\angleOut{0}
    \def\angleIn{180}

    \foreach \i in {1,...,\numNeuronsI}
    {
        \pgfmathsetmacro\posYtmp{0 + \lengthSeparation * (\i - (\numNeuronsI + 1) / 2)}
        \pcbNeuron{neuron-I\i}{
            \posXI, \posYtmp
            }{\scale}{\angleNeuron} ;
    }

    \foreach \i in {1,...,\numNeuronsH}
    {
        \pgfmathsetmacro\posYtmp{0 + \lengthSeparation * (\i - (\numNeuronsH + 1) / 2)}
        \pcbNeuron{neuron-H\i}{
            \posXH, \posYtmp
            }{\scale}{\angleNeuron} ;
    }

    \foreach \i in {1,...,\numNeuronsO}
    {
        \pgfmathsetmacro\posYtmp{0 + \lengthSeparation * (\i - (\numNeuronsO + 1) / 2)}
        \pcbNeuron{neuron-O\i}{
            \posXO, \posYtmp
            }{\scale}{\angleNeuron} ;
    }

    % %%%%%%%%%%%%% connections
    % % input hidden
    \draw[connection] (neuron-I1-Ac) to[out=\angleOut,in=\angleIn] (neuron-H1-Dc);
    \draw[connection] (neuron-I1-Ab) to[out=\angleOut,in=\angleIn] (neuron-H2-Dc);
    \draw[connection] (neuron-I1-Aa) to[out=\angleOut,in=\angleIn] (neuron-H3-Dc);
    %
    \draw[connection] (neuron-I2-Ac) to[out=\angleOut,in=\angleIn] (neuron-H1-Da);
    \draw[connection] (neuron-I2-Ab) to[out=\angleOut,in=\angleIn] (neuron-H2-Da);
    \draw[connection] (neuron-I2-Aa) to[out=\angleOut,in=\angleIn] (neuron-H3-Da);

    % % hidden output
    \draw[connection] (neuron-H1-Ac) to[out=\angleOut,in=\angleIn] (neuron-O1-Da);
    \draw[connection] (neuron-H1-Aa) to[out=\angleOut,in=\angleIn] (neuron-O2-Da);
    %
    \draw[connection] (neuron-H2-Ac) to[out=\angleOut,in=\angleIn] (neuron-O1-Db);
    \draw[connection] (neuron-H2-Aa) to[out=\angleOut,in=\angleIn] (neuron-O2-Db);
    %
    \draw[connection] (neuron-H3-Ac) to[out=\angleOut,in=\angleIn] (neuron-O1-Dc);
    \draw[connection] (neuron-H3-Aa) to[out=\angleOut,in=\angleIn] (neuron-O2-Dc);


\end{tikzpicture}
\end{document}
