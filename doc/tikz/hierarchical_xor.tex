\documentclass[crop,tikz]{standalone}

\usepackage{svg}

\usetikzlibrary{shapes.geometric}
\usetikzlibrary{arrows.meta}
\usetikzlibrary{positioning}
\usetikzlibrary{calc}

\usepackage[outline]{contour}

\usepackage{rotating}

\begin{document}


\definecolor{pcbcolour}{RGB}{90, 32, 90}

\newcommand{\pcbNeuron}[5]{
    % macro for sketch used in overviews
    % arg1: name of nodes
    % arg2: x
    % arg3: y
    % arg4: scale
    % arg5: rotation
    % %%%%%%%%%%%%% draw stuff
    \def\x{#2};
    \def\y{#3}

    \ifthenelse{ \equal{#4}{} }{
        \def\scale{1}
    }{
        \def\scale{#4}
    }
    \ifthenelse{ \equal{#5}{} }{
        \def\rot{0}
    }{
        \def\rot{#5}
    }

    \coordinate (#1) at (\x, \y);

    \node[anchor=center,
          xshift=0.25cm * \scale,yshift=0.1cm * \scale, 
          rotate around={\rot:(#1)},
          ] (whitehead) at (#1)
    {
        % \scaledWidth
        \scalebox{\scale}{
            \includesvg[width=2cm]{neuron_filled.svg}
        }
    };

    % %%%%%%%%%%%%%%%%%%% dendrites
    \pgfmathsetmacro{\lengthDaX}{-0.003cm * \scale}
    \pgfmathsetmacro{\lengthDaY}{0.022cm * \scale}
    \coordinate[rotate around={\rot:(#1)}] (#1-Da) at (\x + \lengthDaX, \y + \lengthDaY);

    \pgfmathsetmacro{\lengthDbX}{-0.023cm * \scale}
    \pgfmathsetmacro{\lengthDbY}{0.005cm * \scale}
    \coordinate[rotate around={\rot:(#1)}]  (#1-Db) at (\x + \lengthDbX, \y + \lengthDbY);

    \pgfmathsetmacro{\lengthDcX}{-0.016cm * \scale}
    \pgfmathsetmacro{\lengthDcY}{-0.0145cm * \scale}
    \coordinate[rotate around={\rot:(#1)}]  (#1-Dc) at (\x + \lengthDcX, \y + \lengthDcY);

    % %%%%%%%%%%%%%%%%%%% axons
    \pgfmathsetmacro{\lengthAxX}{0.0400cm * \scale}
    \pgfmathsetmacro{\lengthAxY}{-0.0130cm * \scale}
    \coordinate[rotate around={\rot:(#1)}]  (#1-Aa) at (\x + \lengthAxX, \y + \lengthAxY);
    \pgfmathsetmacro{\lengthAxX}{0.0395cm * \scale}
    \pgfmathsetmacro{\lengthAxY}{-0.0140cm * \scale}
    \coordinate[rotate around={\rot:(#1)}]  (#1-Ab) at (\x + \lengthAxX, \y + \lengthAxY);
    \pgfmathsetmacro{\lengthAxX}{0.0390cm * \scale}
    \pgfmathsetmacro{\lengthAxY}{-0.0150cm * \scale}
    \coordinate[rotate around={\rot:(#1)}]  (#1-Ac) at (\x + \lengthAxX, \y + \lengthAxY);
}


\begin{tikzpicture}[
    scale=1.,
    >=latex,
    transform shape,
]
    \useasboundingbox[fill=white] (-11, -3.2) rectangle (11, 3.2);

    \def\numNeuronsI{2}
    \def\numNeuronsH{2}
    \def\numNeuronsO{1}
    \def\scale{1.3}

    \def\posXI{-5}
    \def\posXH{-0.25}
    \def\posXO{4.5}

    \def\lengthSeparation{2.8}

    \def\angleNeuron{10}
    \def\angleOut{0}
    \def\angleIn{180}

    \foreach \i in {1,...,\numNeuronsI}
    {
        \pgfmathsetmacro\posYtmp{0 + \lengthSeparation * (\i - (\numNeuronsI + 1) / 2)}
        \pcbNeuron{neuron-I\i}{
            \posXI, \posYtmp
            }{\scale}{\angleNeuron} ;
    }

    \foreach \i in {1,...,\numNeuronsH}
    {
        \pgfmathsetmacro\posYtmp{0 + \lengthSeparation * (\i - (\numNeuronsH + 1) / 2)}
        \pcbNeuron{neuron-H\i}{
            \posXH, \posYtmp
            }{\scale}{\angleNeuron} ;
    }

    \foreach \i in {1,...,\numNeuronsO}
    {
        \pgfmathsetmacro\posYtmp{0 + \lengthSeparation * (\i - (\numNeuronsO + 1) / 2)}
        \pcbNeuron{neuron-O\i}{
            \posXO, \posYtmp
            }{\scale}{\angleNeuron} ;
    }

    % %%%%%%%%%%%%% connections
    % % input hidden
    \draw[connection] (neuron-I1-Ac) to[out=\angleOut,in=\angleIn] (neuron-H1-Dc);
    \draw[connection] (neuron-I1-Ab) to[out=\angleOut,in=\angleIn] (neuron-H2-Dc);
    %
    \draw[connection] (neuron-I2-Ac) to[out=\angleOut,in=\angleIn] (neuron-H1-Da);
    \draw[connection] (neuron-I2-Ab) to[out=\angleOut,in=\angleIn] (neuron-H2-Da);

    % % hidden output
    \draw[connection] (neuron-H1-Ac) to[out=\angleOut,in=\angleIn] (neuron-O1-Dc);
    %
    \draw[connection] (neuron-H2-Ac) to[out=\angleOut,in=\angleIn] (neuron-O1-Da);
    %


\end{tikzpicture}
\end{document}
