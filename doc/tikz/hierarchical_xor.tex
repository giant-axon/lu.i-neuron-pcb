\documentclass[crop,tikz]{standalone}

\usepackage{svg}

\usetikzlibrary{shapes.geometric}
\usetikzlibrary{arrows.meta}
\usetikzlibrary{positioning}
\usetikzlibrary{calc}
\usetikzlibrary{fadings}
\usetikzlibrary{decorations}
\usepgflibrary{decorations.pathmorphing}

\usepackage[outline]{contour}

\usepackage{rotating}

\usepackage{xcolor}

\begin{document}


\definecolor{pcbcolour}{RGB}{90, 32, 90}

\newcommand{\pcbNeuron}[4]{
    % macro for sketch used in overviews
    % arg1: name of nodes
    % arg2: point of node 
    % arg3: argScale
    % arg4: rotation
    % %%%%%%%%%%%%% draw stuff

    \ifthenelse{ \equal{#3}{} }{
        \def\argScale{1}
    }{
        \def\argScale{#3}
    }
    \ifthenelse{ \equal{#4}{} }{
        \def\rot{0}
    }{
        \def\rot{#4}
    }

    \coordinate (#1) at (#2);

    \node[anchor=center,
          xshift=0.25cm * \argScale,yshift=0.1cm * \argScale, 
          rotate around={\rot:(#1)},
          ] (whitehead) at (#1)
    {
        % \scaledWidth
        \scalebox{\argScale}{
            \includesvg[width=2cm]{neuron_filled.svg}
        }
    };

    % %%%%%%%%%%%%%%%%%%% dendrites
    \pgfmathsetmacro{\lengthDaX}{-0.10cm * \argScale}
    \pgfmathsetmacro{\lengthDaY}{0.62cm * \argScale}
    \coordinate[rotate around={\rot:(#1)}] (#1-Da) at ([xshift=\lengthDaX, yshift=\lengthDaY] #1);

    \pgfmathsetmacro{\lengthDbX}{-0.68cm * \argScale}
    \pgfmathsetmacro{\lengthDbY}{0.13cm * \argScale}
    \coordinate[rotate around={\rot:(#1)}]  (#1-Db) at ([xshift=\lengthDbX, yshift=\lengthDbY] #1);

    \pgfmathsetmacro{\lengthDcX}{-0.47cm * \argScale}
    \pgfmathsetmacro{\lengthDcY}{-0.42cm * \argScale}
    \coordinate[rotate around={\rot:(#1)}]  (#1-Dc) at ([xshift=\lengthDcX, yshift=\lengthDcY] #1);

    % %%%%%%%%%%%%%%%%%%% axons
    \pgfmathsetmacro{\lengthAaX}{1.150cm * \argScale}
    \pgfmathsetmacro{\lengthAaY}{-0.360cm * \argScale}
    \coordinate[rotate around={\rot:(#1)}]  (#1-Aa) at ([xshift=\lengthAaX, yshift=\lengthAaY] #1);

    \pgfmathsetmacro{\lengthAbX}{1.135cm * \argScale}
    \pgfmathsetmacro{\lengthAbY}{-0.390cm * \argScale}
    \coordinate[rotate around={\rot:(#1)}]  (#1-Ab) at ([xshift=\lengthAbX, yshift=\lengthAbY] #1);

    \pgfmathsetmacro{\lengthAcX}{1.120cm * \argScale}
    \pgfmathsetmacro{\lengthAcY}{-0.420cm * \argScale}
    \coordinate[rotate around={\rot:(#1)}]  (#1-Ac) at ([xshift=\lengthAcX, yshift=\lengthAcY] #1);
}


\tikzset{
  symbolGround/.pic = {
      \node(-m) [pic actions]{};
      \begin{scope}
          \draw[line cap=round] (0, 0) -- (0, 0.2);
          \draw[line cap=round] (-0.3, 0) -- (0.3, 0);
          \draw[line cap=round] (-0.2, -0.1) -- (0.2, -0.1);
          \draw[line cap=round] (-0.1, -0.2) -- (0.1, -0.2);
      \end{scope}
  },
  symbolMinus/.pic = {
      \node(-m) [pic actions]{};
      \begin{scope}
          \draw[line cap=round] (-0.2, 0) -- (0.2, 0);
      \end{scope}
  },
  symbolPlus/.pic = {
      \node(-m) [pic actions]{};
      \begin{scope}
          \draw[line cap=round] (0, -0.2) -- (0, 0.2);
          \draw[line cap=round] (-0.2, 0) -- (0.2, 0);
      \end{scope}
  }
}

\begin{tikzpicture}[
    scale=1.,
    >=latex,
    transform shape,
]
    \useasboundingbox[fill=white] (-11, -3.2) rectangle (11, 3.2);

    \def\numNeuronsI{2}
    \def\numNeuronsH{2}
    \def\numNeuronsO{1}
    \def\scale{1.7}

    \def\posXI{-7}
    \def\posXH{-0.25}
    \def\posXO{6.5}

    \def\lengthSeparation{2.8}

    \def\angleNeuron{10}
    \def\angleOut{0}
    \def\angleIn{180}

    \foreach \i in {1,...,\numNeuronsI}
    {
        \pgfmathsetmacro\posYtmp{0 + \lengthSeparation * (\i - (\numNeuronsI + 1) / 2)}
        \pcbNeuron{neuron-I\i}{
            \posXI, \posYtmp
            }{\scale}{\angleNeuron} ;
    }

    \foreach \i in {1,...,\numNeuronsH}
    {
        \pgfmathsetmacro\posYtmp{0 + \lengthSeparation * (\i - (\numNeuronsH + 1) / 2)}
        \pcbNeuron{neuron-H\i}{
            \posXH, \posYtmp
            }{\scale}{\angleNeuron} ;
    }

    \foreach \i in {1,...,\numNeuronsO}
    {
        \pgfmathsetmacro\posYtmp{0 + \lengthSeparation * (\i - (\numNeuronsO + 1) / 2)}
        \pcbNeuron{neuron-O\i}{
            \posXO, \posYtmp
            }{\scale}{\angleNeuron} ;
    }

    % %%%%%%%%%%%%% connections
    % % input hidden
    \draw[connection] (neuron-I1-Ac) to[out=\angleOut,in=\angleIn] (neuron-H1-Dc);
    \draw[connection] (neuron-I1-Aa) to[out=\angleOut,in=\angleIn] (neuron-H2-Dc);
    %
    \draw[connection] (neuron-I2-Ac) to[out=\angleOut,in=\angleIn] (neuron-H1-Da);
    \draw[connection] (neuron-I2-Aa) to[out=\angleOut,in=\angleIn] (neuron-H2-Da);

    % % hidden output
    \draw[connection, purple] (neuron-H1-Ab) to[out=\angleOut,in=\angleIn] (neuron-O1-Dc);
    %
    \draw[connection] (neuron-H2-Ab) to[out=\angleOut,in=\angleIn] (neuron-O1-Da);
    %


    % %%%%%%%%%%%%% labels
    \node[neuronlabel] at (neuron-I1-l) {IN 1};
    \node[neuronlabel] at (neuron-I2-l) {IN 2};
    \node[neuronlabel] at (neuron-H1-l) {AND};
    \node[neuronlabel] at (neuron-H2-l) {OR};
    \node[neuronlabel] at (neuron-O1-l) {OUT};

    % %%%%%%%%%%%%% zoom in of synapse
    \coordinate (zoomFocus) at (neuron-O1-Dc);
    \coordinate (zoomedPart) at ([xshift=1.5cm, yshift=-0.9cm] neuron-O1-Dc);
    % ellipses
    \def\zoomEllipseA{0.3 }
    \def\zoomEllipseB{0.2 }
    \def\zoomEllipseAngle{55 }
    \draw[dashed, rotate around={\zoomEllipseAngle: (zoomFocus)}] (zoomFocus) ellipse (\zoomEllipseA cm and \zoomEllipseB cm);
    \coordinate [rotate around={\zoomEllipseAngle: (zoomFocus)}] (ellipseSummit1) at ($(zoomFocus)+(0:\zoomEllipseA cm and \zoomEllipseB cm)$) {};
    \coordinate [rotate around={\zoomEllipseAngle: (zoomFocus)}] (ellipseSummit2) at ($(zoomFocus)+(180:\zoomEllipseA cm and \zoomEllipseB cm)$) {};
    \draw (ellipseSummit1) to ([xshift=9mm, yshift=-3mm]ellipseSummit1);
    \draw (ellipseSummit2) to ([xshift=9mm, yshift=-9mm]ellipseSummit2);
    % pins
    \node[rotate around={145: (zoomedPart)}, 
        shape=rectangle, anchor=center,
        fill=pcbcolour,
        decorate, decoration={zigzag,segment length=2mm, amplitude=.1mm},
        minimum width=6mm, minimum height=10mm,] at (zoomedPart) {};

    \node[rotate around={145: (zoomedPart)}, 
        draw, shape=rectangle, anchor=center,
        preaction={draw,white,line width=2pt, line cap=round},
        minimum width=2.5mm, minimum height=6.5mm,] at (zoomedPart) {};
    
    \node[rotate around={145: (zoomedPart)}, 
        draw, shape=rectangle, anchor=center,
        fill=black, rounded corners=0.8mm,
        minimum width=2mm, minimum height=2mm,] at ([yshift=2mm] zoomedPart) {};
    \node[rotate around={145: (zoomedPart)}, 
        draw, shape=rectangle, anchor=center,
        fill=black, rounded corners=0.8mm,
        minimum width=2mm, minimum height=2mm,] at (zoomedPart) {};
    \node[rotate around={145: (zoomedPart)}, 
        draw, shape=rectangle, anchor=center,
        fill=black, rounded corners=0.8mm,
        minimum width=2mm, minimum height=2mm,] at ([yshift=-2mm] zoomedPart) {};
    % actual pins
    \node[rotate around={145: (zoomedPart)}, 
        draw, shape=rectangle, anchor=center,
        fill=brown, scale=0.3,
        minimum width=2mm, minimum height=2mm,] at ([yshift=-2mm] zoomedPart) {};
    % cables
    \coordinate (cablesMerge) at ([xshift=-1.0cm, yshift=0.5cm]zoomedPart);
    \draw[connectionSmall ] (zoomedPart) to[in=-35, out=145] (cablesMerge);
    \draw[connectionSmall ] ($(zoomedPart)+(235:2mm)$) to[in=-35, out=145] (cablesMerge);


    % silk screen symbols
    \pic [scale=0.17, line width=0.1mm, white, rotate around={55: (zoomedPart)},
        ] at ([yshift=-2.2mm] zoomedPart) {symbolGround};
    \pic [scale=0.17, line width=0.1mm, white, rotate around={55: (zoomedPart)},
        ] at ([xshift=4.2mm] zoomedPart) {symbolPlus};
    \pic [scale=0.17, line width=0.1mm, white, rotate around={55: (zoomedPart)},
        ] at ([xshift=-4.2mm] zoomedPart) {symbolMinus};


\end{tikzpicture}
\end{document}
