\documentclass[convert={density=300,outext=.png}]{standalone}
\usepackage[detect-all]{siunitx}
\DeclareSIUnit{\nothing}{\relax}

\usepackage{circuitikz}
\usetikzlibrary{positioning,fit,shapes.geometric,calc,backgrounds}

\tikzset{
	R/.append style={
		/tikz/circuitikz/bipoles/length=1.1cm
	},
	potentiometer/.append style={
		/tikz/circuitikz/bipoles/length=1.1cm
	},
	C/.append style={
		/tikz/circuitikz/bipoles/length=1.1cm
	},
	Tnmos/.append style={
		/tikz/circuitikz/bipoles/length=1.1cm
	},
	Tpmos/.append style={
		/tikz/circuitikz/bipoles/length=1.1cm
	}
}


\begin{document}

\tikzset{
  symbolGround/.pic = {
      \node(-m) [pic actions]{};
      \begin{scope}
          \draw[line cap=round] (0, 0) -- (0, 0.2);
          \draw[line cap=round] (-0.3, 0) -- (0.3, 0);
          \draw[line cap=round] (-0.2, -0.1) -- (0.2, -0.1);
          \draw[line cap=round] (-0.1, -0.2) -- (0.1, -0.2);
      \end{scope}
  },
  symbolMinus/.pic = {
      \node(-m) [pic actions]{};
      \begin{scope}
          \draw[line cap=round] (-0.2, 0) -- (0.2, 0);
      \end{scope}
  },
  symbolPlus/.pic = {
      \node(-m) [pic actions]{};
      \begin{scope}
          \draw[line cap=round] (0, -0.2) -- (0, 0.2);
          \draw[line cap=round] (-0.2, 0) -- (0.2, 0);
      \end{scope}
  }
}

\begin{tikzpicture}[
    scale=1.,
    >=latex,
    transform shape,
]
    \def\numNeuronsI{2}
    \def\numNeuronsH{2}
    \def\numNeuronsO{1}
    \def\scale{1.7}

    \def\posXI{-7}
    \def\posXH{-0.25}
    \def\posXO{6.5}

    \def\lengthSeparation{2.8}

    \def\angleNeuron{10}
    \def\angleOut{0}
    \def\angleIn{180}

    \foreach \i in {1,...,\numNeuronsI}
    {
        \pgfmathsetmacro\posYtmp{0 + \lengthSeparation * (\i - (\numNeuronsI + 1) / 2)}
        \pcbNeuron{neuron-I\i}{
            \posXI, \posYtmp
        }{\scale}{\angleNeuron}{}{}{tunab};
    }

    \foreach \i in {1,...,\numNeuronsH}
    {
        \pgfmathsetmacro\posYtmp{0 + \lengthSeparation * (\i - (\numNeuronsH + 1) / 2)}
        \pcbNeuron{neuron-H\i}{
            \posXH, \posYtmp
        }{\scale}{\angleNeuron}{}{}{tunab};
    }

    \foreach \i in {1,...,\numNeuronsO}
    {
        \pgfmathsetmacro\posYtmp{0 + \lengthSeparation * (\i - (\numNeuronsO + 1) / 2)}
        \pcbNeuron{neuron-O\i}{
            \posXO, \posYtmp
        }{\scale}{\angleNeuron}{}{}{tunab};
    }

    % %%%%%%%%%%%%% connections
    % % input hidden
    \draw[connection] (neuron-I1-Ac) to[out=\angleOut,in=\angleIn] (neuron-H1-Dc);
    \draw[connection] (neuron-I1-Aa) to[out=\angleOut,in=\angleIn] (neuron-H2-Dc);
    %
    \draw[connection] (neuron-I2-Ac) to[out=\angleOut,in=\angleIn] (neuron-H1-Da);
    \draw[connection] (neuron-I2-Aa) to[out=\angleOut,in=\angleIn] (neuron-H2-Da);

    % % hidden output
    \draw[connection, purple] (neuron-H1-Ab) to[out=\angleOut,in=\angleIn] (neuron-O1-Dc);
    %
    \draw[connection] (neuron-H2-Ab) to[out=\angleOut,in=\angleIn] (neuron-O1-Da);
    %


    % %%%%%%%%%%%%% labels
    \tikzstyle{specialNeuronlabel} = [neuronlabel, fill=pcbcolour, rounded corners=.15cm]
    \node[specialNeuronlabel] at (neuron-I1-l) {IN 1};
    \node[specialNeuronlabel] at (neuron-I2-l) {IN 2};
    \node[specialNeuronlabel] at (neuron-H1-l) {AND};
    \node[specialNeuronlabel] at (neuron-H2-l) {OR};
    \node[specialNeuronlabel] at (neuron-O1-l) {OUT};

    % redraw inhibitory switch
    \node[anchor=north west] (tmp) at ([xshift=0.5cm, yshift=-0.5cm]neuron-O1-tun-WSc) {inhibitory setting};
    \draw[connection] (neuron-O1-tun-WSc) -- (tmp.north west);
	
    \begin{scope}[on background layer]
    	\node[minimum width=20cm, fit=(current bounding box.north west) (current bounding box.south east), inner sep=2pt, fill=white] {};
    \end{scope}
\end{tikzpicture}
\end{document}
