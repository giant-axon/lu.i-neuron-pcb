\documentclass[convert={density=300,outext=.png}]{standalone}
\nonstopmode
\usepackage[detect-all]{siunitx}
\DeclareSIUnit{\nothing}{\relax}

\usepackage{circuitikz}
\usetikzlibrary{positioning,fit,shapes.geometric,calc,backgrounds}

\tikzset{
	R/.append style={
		/tikz/circuitikz/bipoles/length=1.1cm
	},
	potentiometer/.append style={
		/tikz/circuitikz/bipoles/length=1.1cm
	},
	C/.append style={
		/tikz/circuitikz/bipoles/length=1.1cm
	},
	Tnmos/.append style={
		/tikz/circuitikz/bipoles/length=1.1cm
	},
	Tpmos/.append style={
		/tikz/circuitikz/bipoles/length=1.1cm
	}
}


\begin{document}

\begin{tikzpicture}[
    scale=1.,
    >=latex,
    transform shape,
]
    \def\numNeurons{4}
    \def\scale{1.3}
    \def\halfAngleConnection{10}
    \def\angleNeuron{10}
    \def\posXoffsetTop{-10.4}
    \def\posXoffsetMiddle{-8.0}
    \def\posXoffsetBottom{-3.4}
    \def\lengthSeparation{3.5}
    \def\angleOut{-20}
    \def\angleIn{180}


    % % draw ellipse on which the neurons lie
    % \draw (0,0) ellipse (\axlengthA cm and \axlengthB cm );

    % neurons
    \foreach \i in {2,...,\numNeurons}
    {
        % delay Top
        \pcbNeuron{nDelayTop-\i}{
            \posXoffsetTop + \i * \lengthSeparation, 3
        }{\scale}{\angleNeuron} ;
        % delay bottm
        \pgfmathsetmacro{\xPosition}{\posXoffsetBottom + (\numNeurons - \i) * \lengthSeparation}
        \pcbNeuron{nDelayBottom-\i}{
            \xPosition, -3
        }{\scale}{\angleNeuron + 180} ;

        % \node[neuronlabel, font=\sffamily\tiny, ] at (nDelayTop-\i) {nDelayTop-\i};
        % \node[neuronlabel, font=\sffamily\tiny, ] at (nDelayBottom-\i) {nDelayBottom-\i};
    }

    % coincidence
    \foreach \i in {1,...,\numNeurons}
    {
        \pcbNeuron{nCoinci-\i}{
            \posXoffsetMiddle + \i * \lengthSeparation, 0
        }{\scale}{\angleNeuron} ;

        % \node[neuronlabel, font=\sffamily\tiny, ] at (nCoinci-\i) {nCoinci-\i};
    }

    % connections Delay
    \foreach \i in {2,...,\numNeurons}
    {
        \pgfmathtruncatemacro\nexts{ifthenelse(\i==\numNeurons, 1, round(\i+1))}
        \ifthenelse{\equal{\i}{\numNeurons}}{}{
            \draw[connection] (nDelayTop-\i-Ab) to[out=\angleOut,in=\angleIn] (nDelayTop-\nexts-Db);
            \draw[connection] (nDelayBottom-\i-Ab) to[out=\angleOut+180,in=\angleIn+180] (nDelayBottom-\nexts-Db);
        }
    }

    % connections Top to coincidence
    \foreach \i in {1,...,\numNeurons}
    {
        \ifthenelse{\equal{\i}{1}}{}{
            \draw[connection] (nDelayTop-\i-Ac) to[out=\angleOut,in=\angleIn-70] (nCoinci-\i-Da);
        }
        \ifthenelse{\equal{\i}{\numNeurons}}{}{
            \pgfmathtruncatemacro\nMinusI{round(\numNeurons-\i+1)}
            \draw[connection] (nDelayBottom-\nMinusI-Ac) to[out=\angleOut+180,in=\angleIn+40] (nCoinci-\i-Dc);
        }
    }

    % external stimulus

    \coordinate (ears) at (0, -6);
    \coordinate (cableIntermediateBottom) at ([yshift=0cm, xshift=2cm]nDelayBottom-2-Db);
    \coordinate (cableIntermediateTop) at ([yshift=-2cm, xshift=-2cm]nDelayTop-2-Db);

    \draw[connection] ([xshift=1mm]ears) to[out=90, in=-90] (cableIntermediateBottom) to[out=90,in=\angleIn+180] (nDelayBottom-2-Db);
    \draw[connection] (cableIntermediateBottom) to[out=90,in=\angleIn+40] (nCoinci-4-Dc);

    \draw[connection] ([xshift=-1mm]ears) to[out=90, in=-45] ([yshift=-0.5cm, xshift=-2cm]nDelayBottom-\numNeurons) to[out=-45+180, in=-90] (cableIntermediateTop) to[out=90,in=\angleIn] (nDelayTop-2-Db);
    \draw[connection] (cableIntermediateTop) to[out=90,in=\angleIn-70] (nCoinci-1-Da);

    \begin{scope}[on background layer]
    	\node[minimum width=20cm, fit=(current bounding box.north west) (current bounding box.south east), inner sep=2pt, fill=white] {};
    \end{scope}

    \begin{scope}[on background layer]
    	\node[minimum width=20cm, fit=(current bounding box.north west) (current bounding box.south east), inner sep=2pt, fill=white] {};
    \end{scope}
\end{tikzpicture}
\end{document}
