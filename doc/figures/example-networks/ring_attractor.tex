\documentclass[convert={density=300,outext=.png}]{standalone}
\nonstopmode
\usepackage{color}

\usepackage[detect-all]{siunitx}
\DeclareSIUnit{\nothing}{\relax}

\usepackage{circuitikz}
\usetikzlibrary{positioning,fit,shapes.geometric,calc,backgrounds}

\tikzset{
	R/.append style={
		/tikz/circuitikz/bipoles/length=1.1cm
	},
	potentiometer/.append style={
		/tikz/circuitikz/bipoles/length=1.1cm
	},
	C/.append style={
		/tikz/circuitikz/bipoles/length=1.1cm
	},
	Tnmos/.append style={
		/tikz/circuitikz/bipoles/length=1.1cm
	},
	Tpmos/.append style={
		/tikz/circuitikz/bipoles/length=1.1cm
	}
}


\begin{document}

\begin{tikzpicture}[
    scale=1.,
    >=latex,
    transform shape,
    decoration={
        markings,
        mark=at position 0.5 with {\arrow{>}}},
]
    \def\numNeurons{3}
    \def\scale{1.3}
    \def\nOrientationScale{1.6}
    \def\nOrientationRadius{6}
    \def\nOrientationAngle{10}
    \def\nInhibRadius{2.0}
    \def\nInhibAngle{30}
    \def\nSwitchRadius{5.5}
    \def\nSwitchAngle{160}
    \def\nTurnAngle{-20}

    \tikzstyle{signalSelfExc} = [excitatory, solid]
    \tikzstyle{signalToSwitch} = [excitatory, loosely dashdotted]
    \tikzstyle{signalSwitch} = [excitatory, dotted]
    \tikzstyle{signalToInhib} = [excitatory, dashed]
    \tikzstyle{signalTurn} = [excitatory, solid]

    \draw[draw=none, fill=cBGInhibitory!40] (0,0) circle (\nInhibRadius + \scale * 1.4);

    % neurons
    \foreach \i in {1,...,\numNeurons}
    {
        \pgfmathsetmacro\angle{360/\numNeurons * (\i-1) + 90}

        \pcbNeuron{nOrientation-\i}{
            {\angle}:\nOrientationRadius
        }{\nOrientationScale}{\angle + \nOrientationAngle}{}{}{};
        \node[neuronlabel] at (nOrientation-\i) {\textbf{O-\i}};

        \pcbNeuron{nInhib-\i}{
            {\angle}:\nInhibRadius
        }{\scale}{\angle + \nInhibAngle}{}{}{};

        \pcbNeuron{nSwitch-\i}{
            {\angle + 180 / \numNeurons}:\nSwitchRadius
        }{\scale}{\angle + \nSwitchAngle}{}{}{};

    }
    \pcbNeuron{nTurn}{-8, 8}{\scale}{\nTurnAngle}{}{}{};
    \node[neuronlabel, rotate=\nTurnAngle] at (nTurn-l) {turn};



    % connections
    \foreach \i in {1,...,\numNeurons}
    {
        \pgfmathsetmacro\angle{360/\numNeurons * (\i-1) + 90}

        \pgfmathtruncatemacro\iPlusOne{round(Mod(\i + 1 - 1, 3) + 1)}
        \pgfmathtruncatemacro\iPlusTwo{round(Mod(\i + 2 - 1, 3) + 1)}
        \pgfmathsetmacro\anglePlusOne{360/\numNeurons * (\iPlusOne-1) + 90}
        \pgfmathsetmacro\anglePlusTwo{360/\numNeurons * (\iPlusTwo-1) + 90}

        % self connections
        \draw[signalSelfExc, postaction={decorate}] (nOrientation-\i-Aa) to[out=\angle + \nOrientationAngle-30,in=\angle + \nOrientationAngle+180-105] (nOrientation-\i-Da);

        % inhibitory
        \draw[inhibitory, postaction={decorate}] (nInhib-\i-Ab) to[out=\angle + \nInhibAngle,in=\angle + \nOrientationAngle+180] (nOrientation-\i-Db);

        % to inhibitory
        \draw[signalToInhib, postaction={decorate}] (nOrientation-\i-Ac)
            to[out=\angle + \nOrientationAngle-30,
                in=\angle-0] (\angle - 180 / \numNeurons:\nInhibRadius)
            to[out=\angle-0+180,
               in=\anglePlusTwo + \nOrientationAngle+180-70] (nInhib-\iPlusTwo-Da);
        \draw[signalToInhib] (\angle - 180 / \numNeurons:\nInhibRadius)
            to[out=\angle-0+180,
               in=\anglePlusOne + \nOrientationAngle+180-70+ 90] (nInhib-\iPlusOne-Db);

           % switch
        \draw[signalSwitch, postaction={decorate}] (nSwitch-\i-Ab) to[out=\angle + \nOrientationAngle+\nSwitchAngle+180/\numNeurons - 100,in=\angle + \nOrientationAngle-30] (nOrientation-\iPlusOne-Dc);

        % to switch
        \draw[signalToSwitch, postaction={decorate}] (nOrientation-\i-Ab) to[out=\angle + \nOrientationAngle-30,in=\angle + \nOrientationAngle+\nSwitchAngle+180/\numNeurons + 120] (nSwitch-\i-Db);

    }

    % from turn
    \draw[signalTurn, postaction={decorate}] (nTurn-Aa) to[out=\nTurnAngle - 20,in=180] ([yshift=1cm]nOrientation-1-Ab) to[out=0,in=0] (nSwitch-3-Dc);
    \draw[signalTurn, postaction={decorate}] (nTurn-Ab) to[out=\nTurnAngle - 20,in=110]  (nSwitch-1-Dc);
    \def\tmp{120}
    \draw[signalTurn, postaction={decorate}] (nTurn-Ac) to[out=\nTurnAngle - 20,in=\tmp] ([xshift=-1cm]nOrientation-2-Ab) to[out=\tmp+180,in=200] (nSwitch-2-Dc);

    \begin{scope}[on background layer]
    	\node[minimum width=20cm, fit=(current bounding box.north west) (current bounding box.south east), inner sep=2pt,
            fill=white,
            ] {};
    \end{scope}
\end{tikzpicture}
\end{document}
